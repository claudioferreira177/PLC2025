\documentclass[11pt]{article}

    \usepackage[breakable]{tcolorbox}
    \usepackage{parskip} % Stop auto-indenting (to mimic markdown behaviour)
    

    % Basic figure setup, for now with no caption control since it's done
    % automatically by Pandoc (which extracts ![](path) syntax from Markdown).
    \usepackage{graphicx}
    % Keep aspect ratio if custom image width or height is specified
    \setkeys{Gin}{keepaspectratio}
    % Maintain compatibility with old templates. Remove in nbconvert 6.0
    \let\Oldincludegraphics\includegraphics
    % Ensure that by default, figures have no caption (until we provide a
    % proper Figure object with a Caption API and a way to capture that
    % in the conversion process - todo).
    \usepackage{caption}
    \DeclareCaptionFormat{nocaption}{}
    \captionsetup{format=nocaption,aboveskip=0pt,belowskip=0pt}

    \usepackage{float}
    \floatplacement{figure}{H} % forces figures to be placed at the correct location
    \usepackage{xcolor} % Allow colors to be defined
    \usepackage{enumerate} % Needed for markdown enumerations to work
    \usepackage{geometry} % Used to adjust the document margins
    \usepackage{amsmath} % Equations
    \usepackage{amssymb} % Equations
    \usepackage{textcomp} % defines textquotesingle
    % Hack from http://tex.stackexchange.com/a/47451/13684:
    \AtBeginDocument{%
        \def\PYZsq{\textquotesingle}% Upright quotes in Pygmentized code
    }
    \usepackage{upquote} % Upright quotes for verbatim code
    \usepackage{eurosym} % defines \euro

    \usepackage{iftex}
    \ifPDFTeX
        \usepackage[T1]{fontenc}
        \IfFileExists{alphabeta.sty}{
              \usepackage{alphabeta}
          }{
              \usepackage[mathletters]{ucs}
              \usepackage[utf8x]{inputenc}
          }
    \else
        \usepackage{fontspec}
        \usepackage{unicode-math}
    \fi

    \usepackage{fancyvrb} % verbatim replacement that allows latex
    \usepackage{grffile} % extends the file name processing of package graphics
                         % to support a larger range
    \makeatletter % fix for old versions of grffile with XeLaTeX
    \@ifpackagelater{grffile}{2019/11/01}
    {
      % Do nothing on new versions
    }
    {
      \def\Gread@@xetex#1{%
        \IfFileExists{"\Gin@base".bb}%
        {\Gread@eps{\Gin@base.bb}}%
        {\Gread@@xetex@aux#1}%
      }
    }
    \makeatother
    \usepackage[Export]{adjustbox} % Used to constrain images to a maximum size
    \adjustboxset{max size={0.9\linewidth}{0.9\paperheight}}

    % The hyperref package gives us a pdf with properly built
    % internal navigation ('pdf bookmarks' for the table of contents,
    % internal cross-reference links, web links for URLs, etc.)
    \usepackage{hyperref}
    % The default LaTeX title has an obnoxious amount of whitespace. By default,
    % titling removes some of it. It also provides customization options.
    \usepackage{titling}
    \usepackage{longtable} % longtable support required by pandoc >1.10
    \usepackage{booktabs}  % table support for pandoc > 1.12.2
    \usepackage{array}     % table support for pandoc >= 2.11.3
    \usepackage{calc}      % table minipage width calculation for pandoc >= 2.11.1
    \usepackage[inline]{enumitem} % IRkernel/repr support (it uses the enumerate* environment)
    \usepackage[normalem]{ulem} % ulem is needed to support strikethroughs (\sout)
                                % normalem makes italics be italics, not underlines
    \usepackage{soul}      % strikethrough (\st) support for pandoc >= 3.0.0
    \usepackage{mathrsfs}
    

    
    % Colors for the hyperref package
    \definecolor{urlcolor}{rgb}{0,.145,.698}
    \definecolor{linkcolor}{rgb}{.71,0.21,0.01}
    \definecolor{citecolor}{rgb}{.12,.54,.11}

    % ANSI colors
    \definecolor{ansi-black}{HTML}{3E424D}
    \definecolor{ansi-black-intense}{HTML}{282C36}
    \definecolor{ansi-red}{HTML}{E75C58}
    \definecolor{ansi-red-intense}{HTML}{B22B31}
    \definecolor{ansi-green}{HTML}{00A250}
    \definecolor{ansi-green-intense}{HTML}{007427}
    \definecolor{ansi-yellow}{HTML}{DDB62B}
    \definecolor{ansi-yellow-intense}{HTML}{B27D12}
    \definecolor{ansi-blue}{HTML}{208FFB}
    \definecolor{ansi-blue-intense}{HTML}{0065CA}
    \definecolor{ansi-magenta}{HTML}{D160C4}
    \definecolor{ansi-magenta-intense}{HTML}{A03196}
    \definecolor{ansi-cyan}{HTML}{60C6C8}
    \definecolor{ansi-cyan-intense}{HTML}{258F8F}
    \definecolor{ansi-white}{HTML}{C5C1B4}
    \definecolor{ansi-white-intense}{HTML}{A1A6B2}
    \definecolor{ansi-default-inverse-fg}{HTML}{FFFFFF}
    \definecolor{ansi-default-inverse-bg}{HTML}{000000}

    % common color for the border for error outputs.
    \definecolor{outerrorbackground}{HTML}{FFDFDF}

    % commands and environments needed by pandoc snippets
    % extracted from the output of `pandoc -s`
    \providecommand{\tightlist}{%
      \setlength{\itemsep}{0pt}\setlength{\parskip}{0pt}}
    \DefineVerbatimEnvironment{Highlighting}{Verbatim}{commandchars=\\\{\}}
    % Add ',fontsize=\small' for more characters per line
    \newenvironment{Shaded}{}{}
    \newcommand{\KeywordTok}[1]{\textcolor[rgb]{0.00,0.44,0.13}{\textbf{{#1}}}}
    \newcommand{\DataTypeTok}[1]{\textcolor[rgb]{0.56,0.13,0.00}{{#1}}}
    \newcommand{\DecValTok}[1]{\textcolor[rgb]{0.25,0.63,0.44}{{#1}}}
    \newcommand{\BaseNTok}[1]{\textcolor[rgb]{0.25,0.63,0.44}{{#1}}}
    \newcommand{\FloatTok}[1]{\textcolor[rgb]{0.25,0.63,0.44}{{#1}}}
    \newcommand{\CharTok}[1]{\textcolor[rgb]{0.25,0.44,0.63}{{#1}}}
    \newcommand{\StringTok}[1]{\textcolor[rgb]{0.25,0.44,0.63}{{#1}}}
    \newcommand{\CommentTok}[1]{\textcolor[rgb]{0.38,0.63,0.69}{\textit{{#1}}}}
    \newcommand{\OtherTok}[1]{\textcolor[rgb]{0.00,0.44,0.13}{{#1}}}
    \newcommand{\AlertTok}[1]{\textcolor[rgb]{1.00,0.00,0.00}{\textbf{{#1}}}}
    \newcommand{\FunctionTok}[1]{\textcolor[rgb]{0.02,0.16,0.49}{{#1}}}
    \newcommand{\RegionMarkerTok}[1]{{#1}}
    \newcommand{\ErrorTok}[1]{\textcolor[rgb]{1.00,0.00,0.00}{\textbf{{#1}}}}
    \newcommand{\NormalTok}[1]{{#1}}

    % Additional commands for more recent versions of Pandoc
    \newcommand{\ConstantTok}[1]{\textcolor[rgb]{0.53,0.00,0.00}{{#1}}}
    \newcommand{\SpecialCharTok}[1]{\textcolor[rgb]{0.25,0.44,0.63}{{#1}}}
    \newcommand{\VerbatimStringTok}[1]{\textcolor[rgb]{0.25,0.44,0.63}{{#1}}}
    \newcommand{\SpecialStringTok}[1]{\textcolor[rgb]{0.73,0.40,0.53}{{#1}}}
    \newcommand{\ImportTok}[1]{{#1}}
    \newcommand{\DocumentationTok}[1]{\textcolor[rgb]{0.73,0.13,0.13}{\textit{{#1}}}}
    \newcommand{\AnnotationTok}[1]{\textcolor[rgb]{0.38,0.63,0.69}{\textbf{\textit{{#1}}}}}
    \newcommand{\CommentVarTok}[1]{\textcolor[rgb]{0.38,0.63,0.69}{\textbf{\textit{{#1}}}}}
    \newcommand{\VariableTok}[1]{\textcolor[rgb]{0.10,0.09,0.49}{{#1}}}
    \newcommand{\ControlFlowTok}[1]{\textcolor[rgb]{0.00,0.44,0.13}{\textbf{{#1}}}}
    \newcommand{\OperatorTok}[1]{\textcolor[rgb]{0.40,0.40,0.40}{{#1}}}
    \newcommand{\BuiltInTok}[1]{{#1}}
    \newcommand{\ExtensionTok}[1]{{#1}}
    \newcommand{\PreprocessorTok}[1]{\textcolor[rgb]{0.74,0.48,0.00}{{#1}}}
    \newcommand{\AttributeTok}[1]{\textcolor[rgb]{0.49,0.56,0.16}{{#1}}}
    \newcommand{\InformationTok}[1]{\textcolor[rgb]{0.38,0.63,0.69}{\textbf{\textit{{#1}}}}}
    \newcommand{\WarningTok}[1]{\textcolor[rgb]{0.38,0.63,0.69}{\textbf{\textit{{#1}}}}}
    \makeatletter
    \newsavebox\pandoc@box
    \newcommand*\pandocbounded[1]{%
      \sbox\pandoc@box{#1}%
      % scaling factors for width and height
      \Gscale@div\@tempa\textheight{\dimexpr\ht\pandoc@box+\dp\pandoc@box\relax}%
      \Gscale@div\@tempb\linewidth{\wd\pandoc@box}%
      % select the smaller of both
      \ifdim\@tempb\p@<\@tempa\p@
        \let\@tempa\@tempb
      \fi
      % scaling accordingly (\@tempa < 1)
      \ifdim\@tempa\p@<\p@
        \scalebox{\@tempa}{\usebox\pandoc@box}%
      % scaling not needed, use as it is
      \else
        \usebox{\pandoc@box}%
      \fi
    }
    \makeatother

    % Define a nice break command that doesn't care if a line doesn't already
    % exist.
    \def\br{\hspace*{\fill} \\* }
    % Math Jax compatibility definitions
    \def\gt{>}
    \def\lt{<}
    \let\Oldtex\TeX
    \let\Oldlatex\LaTeX
    \renewcommand{\TeX}{\textrm{\Oldtex}}
    \renewcommand{\LaTeX}{\textrm{\Oldlatex}}
    % Document parameters
    % Document title
    \title{Relatorio\_final}
    
    
    
    
    
    
    
% Pygments definitions
\makeatletter
\def\PY@reset{\let\PY@it=\relax \let\PY@bf=\relax%
    \let\PY@ul=\relax \let\PY@tc=\relax%
    \let\PY@bc=\relax \let\PY@ff=\relax}
\def\PY@tok#1{\csname PY@tok@#1\endcsname}
\def\PY@toks#1+{\ifx\relax#1\empty\else%
    \PY@tok{#1}\expandafter\PY@toks\fi}
\def\PY@do#1{\PY@bc{\PY@tc{\PY@ul{%
    \PY@it{\PY@bf{\PY@ff{#1}}}}}}}
\def\PY#1#2{\PY@reset\PY@toks#1+\relax+\PY@do{#2}}

\@namedef{PY@tok@w}{\def\PY@tc##1{\textcolor[rgb]{0.73,0.73,0.73}{##1}}}
\@namedef{PY@tok@c}{\let\PY@it=\textit\def\PY@tc##1{\textcolor[rgb]{0.24,0.48,0.48}{##1}}}
\@namedef{PY@tok@cp}{\def\PY@tc##1{\textcolor[rgb]{0.61,0.40,0.00}{##1}}}
\@namedef{PY@tok@k}{\let\PY@bf=\textbf\def\PY@tc##1{\textcolor[rgb]{0.00,0.50,0.00}{##1}}}
\@namedef{PY@tok@kp}{\def\PY@tc##1{\textcolor[rgb]{0.00,0.50,0.00}{##1}}}
\@namedef{PY@tok@kt}{\def\PY@tc##1{\textcolor[rgb]{0.69,0.00,0.25}{##1}}}
\@namedef{PY@tok@o}{\def\PY@tc##1{\textcolor[rgb]{0.40,0.40,0.40}{##1}}}
\@namedef{PY@tok@ow}{\let\PY@bf=\textbf\def\PY@tc##1{\textcolor[rgb]{0.67,0.13,1.00}{##1}}}
\@namedef{PY@tok@nb}{\def\PY@tc##1{\textcolor[rgb]{0.00,0.50,0.00}{##1}}}
\@namedef{PY@tok@nf}{\def\PY@tc##1{\textcolor[rgb]{0.00,0.00,1.00}{##1}}}
\@namedef{PY@tok@nc}{\let\PY@bf=\textbf\def\PY@tc##1{\textcolor[rgb]{0.00,0.00,1.00}{##1}}}
\@namedef{PY@tok@nn}{\let\PY@bf=\textbf\def\PY@tc##1{\textcolor[rgb]{0.00,0.00,1.00}{##1}}}
\@namedef{PY@tok@ne}{\let\PY@bf=\textbf\def\PY@tc##1{\textcolor[rgb]{0.80,0.25,0.22}{##1}}}
\@namedef{PY@tok@nv}{\def\PY@tc##1{\textcolor[rgb]{0.10,0.09,0.49}{##1}}}
\@namedef{PY@tok@no}{\def\PY@tc##1{\textcolor[rgb]{0.53,0.00,0.00}{##1}}}
\@namedef{PY@tok@nl}{\def\PY@tc##1{\textcolor[rgb]{0.46,0.46,0.00}{##1}}}
\@namedef{PY@tok@ni}{\let\PY@bf=\textbf\def\PY@tc##1{\textcolor[rgb]{0.44,0.44,0.44}{##1}}}
\@namedef{PY@tok@na}{\def\PY@tc##1{\textcolor[rgb]{0.41,0.47,0.13}{##1}}}
\@namedef{PY@tok@nt}{\let\PY@bf=\textbf\def\PY@tc##1{\textcolor[rgb]{0.00,0.50,0.00}{##1}}}
\@namedef{PY@tok@nd}{\def\PY@tc##1{\textcolor[rgb]{0.67,0.13,1.00}{##1}}}
\@namedef{PY@tok@s}{\def\PY@tc##1{\textcolor[rgb]{0.73,0.13,0.13}{##1}}}
\@namedef{PY@tok@sd}{\let\PY@it=\textit\def\PY@tc##1{\textcolor[rgb]{0.73,0.13,0.13}{##1}}}
\@namedef{PY@tok@si}{\let\PY@bf=\textbf\def\PY@tc##1{\textcolor[rgb]{0.64,0.35,0.47}{##1}}}
\@namedef{PY@tok@se}{\let\PY@bf=\textbf\def\PY@tc##1{\textcolor[rgb]{0.67,0.36,0.12}{##1}}}
\@namedef{PY@tok@sr}{\def\PY@tc##1{\textcolor[rgb]{0.64,0.35,0.47}{##1}}}
\@namedef{PY@tok@ss}{\def\PY@tc##1{\textcolor[rgb]{0.10,0.09,0.49}{##1}}}
\@namedef{PY@tok@sx}{\def\PY@tc##1{\textcolor[rgb]{0.00,0.50,0.00}{##1}}}
\@namedef{PY@tok@m}{\def\PY@tc##1{\textcolor[rgb]{0.40,0.40,0.40}{##1}}}
\@namedef{PY@tok@gh}{\let\PY@bf=\textbf\def\PY@tc##1{\textcolor[rgb]{0.00,0.00,0.50}{##1}}}
\@namedef{PY@tok@gu}{\let\PY@bf=\textbf\def\PY@tc##1{\textcolor[rgb]{0.50,0.00,0.50}{##1}}}
\@namedef{PY@tok@gd}{\def\PY@tc##1{\textcolor[rgb]{0.63,0.00,0.00}{##1}}}
\@namedef{PY@tok@gi}{\def\PY@tc##1{\textcolor[rgb]{0.00,0.52,0.00}{##1}}}
\@namedef{PY@tok@gr}{\def\PY@tc##1{\textcolor[rgb]{0.89,0.00,0.00}{##1}}}
\@namedef{PY@tok@ge}{\let\PY@it=\textit}
\@namedef{PY@tok@gs}{\let\PY@bf=\textbf}
\@namedef{PY@tok@ges}{\let\PY@bf=\textbf\let\PY@it=\textit}
\@namedef{PY@tok@gp}{\let\PY@bf=\textbf\def\PY@tc##1{\textcolor[rgb]{0.00,0.00,0.50}{##1}}}
\@namedef{PY@tok@go}{\def\PY@tc##1{\textcolor[rgb]{0.44,0.44,0.44}{##1}}}
\@namedef{PY@tok@gt}{\def\PY@tc##1{\textcolor[rgb]{0.00,0.27,0.87}{##1}}}
\@namedef{PY@tok@err}{\def\PY@bc##1{{\setlength{\fboxsep}{\string -\fboxrule}\fcolorbox[rgb]{1.00,0.00,0.00}{1,1,1}{\strut ##1}}}}
\@namedef{PY@tok@kc}{\let\PY@bf=\textbf\def\PY@tc##1{\textcolor[rgb]{0.00,0.50,0.00}{##1}}}
\@namedef{PY@tok@kd}{\let\PY@bf=\textbf\def\PY@tc##1{\textcolor[rgb]{0.00,0.50,0.00}{##1}}}
\@namedef{PY@tok@kn}{\let\PY@bf=\textbf\def\PY@tc##1{\textcolor[rgb]{0.00,0.50,0.00}{##1}}}
\@namedef{PY@tok@kr}{\let\PY@bf=\textbf\def\PY@tc##1{\textcolor[rgb]{0.00,0.50,0.00}{##1}}}
\@namedef{PY@tok@bp}{\def\PY@tc##1{\textcolor[rgb]{0.00,0.50,0.00}{##1}}}
\@namedef{PY@tok@fm}{\def\PY@tc##1{\textcolor[rgb]{0.00,0.00,1.00}{##1}}}
\@namedef{PY@tok@vc}{\def\PY@tc##1{\textcolor[rgb]{0.10,0.09,0.49}{##1}}}
\@namedef{PY@tok@vg}{\def\PY@tc##1{\textcolor[rgb]{0.10,0.09,0.49}{##1}}}
\@namedef{PY@tok@vi}{\def\PY@tc##1{\textcolor[rgb]{0.10,0.09,0.49}{##1}}}
\@namedef{PY@tok@vm}{\def\PY@tc##1{\textcolor[rgb]{0.10,0.09,0.49}{##1}}}
\@namedef{PY@tok@sa}{\def\PY@tc##1{\textcolor[rgb]{0.73,0.13,0.13}{##1}}}
\@namedef{PY@tok@sb}{\def\PY@tc##1{\textcolor[rgb]{0.73,0.13,0.13}{##1}}}
\@namedef{PY@tok@sc}{\def\PY@tc##1{\textcolor[rgb]{0.73,0.13,0.13}{##1}}}
\@namedef{PY@tok@dl}{\def\PY@tc##1{\textcolor[rgb]{0.73,0.13,0.13}{##1}}}
\@namedef{PY@tok@s2}{\def\PY@tc##1{\textcolor[rgb]{0.73,0.13,0.13}{##1}}}
\@namedef{PY@tok@sh}{\def\PY@tc##1{\textcolor[rgb]{0.73,0.13,0.13}{##1}}}
\@namedef{PY@tok@s1}{\def\PY@tc##1{\textcolor[rgb]{0.73,0.13,0.13}{##1}}}
\@namedef{PY@tok@mb}{\def\PY@tc##1{\textcolor[rgb]{0.40,0.40,0.40}{##1}}}
\@namedef{PY@tok@mf}{\def\PY@tc##1{\textcolor[rgb]{0.40,0.40,0.40}{##1}}}
\@namedef{PY@tok@mh}{\def\PY@tc##1{\textcolor[rgb]{0.40,0.40,0.40}{##1}}}
\@namedef{PY@tok@mi}{\def\PY@tc##1{\textcolor[rgb]{0.40,0.40,0.40}{##1}}}
\@namedef{PY@tok@il}{\def\PY@tc##1{\textcolor[rgb]{0.40,0.40,0.40}{##1}}}
\@namedef{PY@tok@mo}{\def\PY@tc##1{\textcolor[rgb]{0.40,0.40,0.40}{##1}}}
\@namedef{PY@tok@ch}{\let\PY@it=\textit\def\PY@tc##1{\textcolor[rgb]{0.24,0.48,0.48}{##1}}}
\@namedef{PY@tok@cm}{\let\PY@it=\textit\def\PY@tc##1{\textcolor[rgb]{0.24,0.48,0.48}{##1}}}
\@namedef{PY@tok@cpf}{\let\PY@it=\textit\def\PY@tc##1{\textcolor[rgb]{0.24,0.48,0.48}{##1}}}
\@namedef{PY@tok@c1}{\let\PY@it=\textit\def\PY@tc##1{\textcolor[rgb]{0.24,0.48,0.48}{##1}}}
\@namedef{PY@tok@cs}{\let\PY@it=\textit\def\PY@tc##1{\textcolor[rgb]{0.24,0.48,0.48}{##1}}}

\def\PYZbs{\char`\\}
\def\PYZus{\char`\_}
\def\PYZob{\char`\{}
\def\PYZcb{\char`\}}
\def\PYZca{\char`\^}
\def\PYZam{\char`\&}
\def\PYZlt{\char`\<}
\def\PYZgt{\char`\>}
\def\PYZsh{\char`\#}
\def\PYZpc{\char`\%}
\def\PYZdl{\char`\$}
\def\PYZhy{\char`\-}
\def\PYZsq{\char`\'}
\def\PYZdq{\char`\"}
\def\PYZti{\char`\~}
% for compatibility with earlier versions
\def\PYZat{@}
\def\PYZlb{[}
\def\PYZrb{]}
\makeatother


    % For linebreaks inside Verbatim environment from package fancyvrb.
    \makeatletter
        \newbox\Wrappedcontinuationbox
        \newbox\Wrappedvisiblespacebox
        \newcommand*\Wrappedvisiblespace {\textcolor{red}{\textvisiblespace}}
        \newcommand*\Wrappedcontinuationsymbol {\textcolor{red}{\llap{\tiny$\m@th\hookrightarrow$}}}
        \newcommand*\Wrappedcontinuationindent {3ex }
        \newcommand*\Wrappedafterbreak {\kern\Wrappedcontinuationindent\copy\Wrappedcontinuationbox}
        % Take advantage of the already applied Pygments mark-up to insert
        % potential linebreaks for TeX processing.
        %        {, <, #, %, $, ' and ": go to next line.
        %        _, }, ^, &, >, - and ~: stay at end of broken line.
        % Use of \textquotesingle for straight quote.
        \newcommand*\Wrappedbreaksatspecials {%
            \def\PYGZus{\discretionary{\char`\_}{\Wrappedafterbreak}{\char`\_}}%
            \def\PYGZob{\discretionary{}{\Wrappedafterbreak\char`\{}{\char`\{}}%
            \def\PYGZcb{\discretionary{\char`\}}{\Wrappedafterbreak}{\char`\}}}%
            \def\PYGZca{\discretionary{\char`\^}{\Wrappedafterbreak}{\char`\^}}%
            \def\PYGZam{\discretionary{\char`\&}{\Wrappedafterbreak}{\char`\&}}%
            \def\PYGZlt{\discretionary{}{\Wrappedafterbreak\char`\<}{\char`\<}}%
            \def\PYGZgt{\discretionary{\char`\>}{\Wrappedafterbreak}{\char`\>}}%
            \def\PYGZsh{\discretionary{}{\Wrappedafterbreak\char`\#}{\char`\#}}%
            \def\PYGZpc{\discretionary{}{\Wrappedafterbreak\char`\%}{\char`\%}}%
            \def\PYGZdl{\discretionary{}{\Wrappedafterbreak\char`\$}{\char`\$}}%
            \def\PYGZhy{\discretionary{\char`\-}{\Wrappedafterbreak}{\char`\-}}%
            \def\PYGZsq{\discretionary{}{\Wrappedafterbreak\textquotesingle}{\textquotesingle}}%
            \def\PYGZdq{\discretionary{}{\Wrappedafterbreak\char`\"}{\char`\"}}%
            \def\PYGZti{\discretionary{\char`\~}{\Wrappedafterbreak}{\char`\~}}%
        }
        % Some characters . , ; ? ! / are not pygmentized.
        % This macro makes them "active" and they will insert potential linebreaks
        \newcommand*\Wrappedbreaksatpunct {%
            \lccode`\~`\.\lowercase{\def~}{\discretionary{\hbox{\char`\.}}{\Wrappedafterbreak}{\hbox{\char`\.}}}%
            \lccode`\~`\,\lowercase{\def~}{\discretionary{\hbox{\char`\,}}{\Wrappedafterbreak}{\hbox{\char`\,}}}%
            \lccode`\~`\;\lowercase{\def~}{\discretionary{\hbox{\char`\;}}{\Wrappedafterbreak}{\hbox{\char`\;}}}%
            \lccode`\~`\:\lowercase{\def~}{\discretionary{\hbox{\char`\:}}{\Wrappedafterbreak}{\hbox{\char`\:}}}%
            \lccode`\~`\?\lowercase{\def~}{\discretionary{\hbox{\char`\?}}{\Wrappedafterbreak}{\hbox{\char`\?}}}%
            \lccode`\~`\!\lowercase{\def~}{\discretionary{\hbox{\char`\!}}{\Wrappedafterbreak}{\hbox{\char`\!}}}%
            \lccode`\~`\/\lowercase{\def~}{\discretionary{\hbox{\char`\/}}{\Wrappedafterbreak}{\hbox{\char`\/}}}%
            \catcode`\.\active
            \catcode`\,\active
            \catcode`\;\active
            \catcode`\:\active
            \catcode`\?\active
            \catcode`\!\active
            \catcode`\/\active
            \lccode`\~`\~
        }
    \makeatother

    \let\OriginalVerbatim=\Verbatim
    \makeatletter
    \renewcommand{\Verbatim}[1][1]{%
        %\parskip\z@skip
        \sbox\Wrappedcontinuationbox {\Wrappedcontinuationsymbol}%
        \sbox\Wrappedvisiblespacebox {\FV@SetupFont\Wrappedvisiblespace}%
        \def\FancyVerbFormatLine ##1{\hsize\linewidth
            \vtop{\raggedright\hyphenpenalty\z@\exhyphenpenalty\z@
                \doublehyphendemerits\z@\finalhyphendemerits\z@
                \strut ##1\strut}%
        }%
        % If the linebreak is at a space, the latter will be displayed as visible
        % space at end of first line, and a continuation symbol starts next line.
        % Stretch/shrink are however usually zero for typewriter font.
        \def\FV@Space {%
            \nobreak\hskip\z@ plus\fontdimen3\font minus\fontdimen4\font
            \discretionary{\copy\Wrappedvisiblespacebox}{\Wrappedafterbreak}
            {\kern\fontdimen2\font}%
        }%

        % Allow breaks at special characters using \PYG... macros.
        \Wrappedbreaksatspecials
        % Breaks at punctuation characters . , ; ? ! and / need catcode=\active
        \OriginalVerbatim[#1,codes*=\Wrappedbreaksatpunct]%
    }
    \makeatother

    % Exact colors from NB
    \definecolor{incolor}{HTML}{303F9F}
    \definecolor{outcolor}{HTML}{D84315}
    \definecolor{cellborder}{HTML}{CFCFCF}
    \definecolor{cellbackground}{HTML}{F7F7F7}

    % prompt
    \makeatletter
    \newcommand{\boxspacing}{\kern\kvtcb@left@rule\kern\kvtcb@boxsep}
    \makeatother
    \newcommand{\prompt}[4]{
        {\ttfamily\llap{{\color{#2}[#3]:\hspace{3pt}#4}}\vspace{-\baselineskip}}
    }
    

    
    % Prevent overflowing lines due to hard-to-break entities
    \sloppy
    % Setup hyperref package
    \hypersetup{
      breaklinks=true,  % so long urls are correctly broken across lines
      colorlinks=true,
      urlcolor=urlcolor,
      linkcolor=linkcolor,
      citecolor=citecolor,
      }
    % Slightly bigger margins than the latex defaults
    
    \geometry{verbose,tmargin=1in,bmargin=1in,lmargin=1in,rmargin=1in}
    
    

\begin{document}
    
    \maketitle
    
    

    
    \hypertarget{processamento-de-linguagens-e-compiladores---trabalho-pruxe1tico}{%
\section{Processamento de Linguagens e Compiladores - Trabalho
Prático}\label{processamento-de-linguagens-e-compiladores---trabalho-pruxe1tico}}

\textbf{Autores.} - \emph{Cláudio Rafael Oliveira Ferreira} (A108577) -
\emph{Marco António Ferreira Abreu} (A108578) - \emph{Nelson Daniel
Araújo Sousa} (A109068)

\textbf{Linguagem:} Python

\textbf{Estrutura}: 1) Introdução e Arquitetura do Sistema 2) Análise
Léxica 3) Análise Sintática 4) Análise Semântica e Gestão de Contexto 5)
Geração de Código 6) Testes e Resultados 7) Conclusão e Dificuldades
Superadas

    \hypertarget{introduuxe7uxe3o-e-arquitetura-do-sistema}{%
\section{1. Introdução e Arquitetura do
Sistema}\label{introduuxe7uxe3o-e-arquitetura-do-sistema}}

O objetivo deste projeto é o desenvolvimento de um compilador robusto
para a linguagem \textbf{Pascal Standard}. O sistema foi desenhado para
realizar a tradução completa de programas fonte para uma representação
de baixo nível, especificamente para o conjunto de instruções da
\textbf{Máquina Virtual (VM)} de stack disponibilizada.

\hypertarget{metodologia-de-desenvolvimento}{%
\subsection{1.1. Metodologia de
Desenvolvimento}\label{metodologia-de-desenvolvimento}}

A implementação baseia-se num pipeline de múltiplas passagens
(multi-pass compiler), o que garante uma separação clara entre a análise
linguística e a síntese de código. Esta abordagem facilita a depuração e
permite que cada fase do compilador valide a integridade dos dados antes
de os passar à etapa seguinte.

\hypertarget{estrutura-modular-e-organizauxe7uxe3o}{%
\subsection{1.2. Estrutura Modular e
Organização}\label{estrutura-modular-e-organizauxe7uxe3o}}

O projeto foi organizado de forma estrita para garantir escalabilidade e
permitir a automação de testes. A estrutura de pastas reflete esta
organização:

\begin{itemize}
\tightlist
\item
  \textbf{main.py:} Interface de linha de comando (CLI) que serve como
  ponto de entrada para a compilação de ficheiros individuais.
\item
  \textbf{/src:} Núcleo do compilador, onde reside a lógica dividida por
  módulos:

  \begin{itemize}
  \tightlist
  \item
    \textbf{pascal\_analex.py \& parser.py:} Responsáveis pelo
    \textbf{Front-end} (Análise Léxica e Sintática).
  \item
    \textbf{sem.py \& context.py:} Responsáveis pelo \textbf{Middle-end}
    (Análise Semântica, Gestão de Escopos e Tabelas de Símbolos).
  \item
    \textbf{codegen.py:} Responsável pelo \textbf{Back-end} (Geração de
    código para a VM).
  \end{itemize}
\item
  \textbf{/tests:} Infraestrutura avançada de testes automatizados:

  \begin{itemize}
  \tightlist
  \item
    \textbf{run\_tests.py:} Motor de execução que automatiza a validação
    de casos de sucesso e de erro.
  \item
    \textbf{/cases:} Subdividida em \textbf{ok/} (programas válidos) e
    \textbf{error/} (programas com erros propositados para teste de
    robustez).
  \item
    \textbf{/manifests:} Contém o ficheiro \textbf{error\_cases.json},
    que define as mensagens de erro esperadas para cada teste negativo.
    \textbf{/out\_vm:} Diretoria de destino para os ficheiros de
    assembly (\textbf{.vm}) resultantes da compilação.
  \end{itemize}
\end{itemize}

\hypertarget{fluxo-de-trabalho-pipeline}{%
\subsection{1.3. Fluxo de Trabalho
(Pipeline)}\label{fluxo-de-trabalho-pipeline}}

O processo de compilação segue um fluxo sequencial e rigoroso: 1)
\textbf{Leitura e Tokenização:} O ficheiro \textbf{.pas} é lido e o
Lexer converte o texto num fluxo de tokens. 2) \textbf{Análise e
Validação:} O Parser valida a estrutura gramatical e invoca o Analisador
Semântico para verificar tipos e declarações. 3) \textbf{Gestão de
Contexto:} Durante a análise, o \textbf{CompilerContext} gere a alocação
de memória (endereços globais e locais) e a visibilidade de variáveis.
4) \textbf{Emissão de Código:} Em caso de sucesso, o código assembly é
gerado, otimizando a stack discipline (especialmente em subprogramas).
5) \textbf{Interrupção por Erro:} Caso ocorra uma falha em qualquer
fase, o compilador interrompe imediatamente o processo, emitindo uma
mensagem de erro informativa (ex: \textbf{SemanticError} ou
\textbf{SyntaxParseError}) que indica a linha e a causa do problema.

    \hypertarget{anuxe1lise-luxe9xica-srcpascal_analex.py}{%
\section{2. Análise Léxica
(src/pascal\_analex.py)}\label{anuxe1lise-luxe9xica-srcpascal_analex.py}}

A análise léxica é a primeira etapa do processo de compilação. O seu
objetivo é ler o fluxo de caracteres do código-fonte e agrupá-los em
unidades significativas chamadas \textbf{tokens}.

\hypertarget{ferramentas-e-configurauxe7uxe3o}{%
\subsection{2.1. Ferramentas e
Configuração}\label{ferramentas-e-configurauxe7uxe3o}}

Para esta fase, utilizámos a biblioteca \textbf{ply.lex}. A configuração
baseia-se na definição de uma lista de tokens e na utilização de
expressões regulares para descrever o padrão de cada um deles. O
analisador foi desenhado para ser \textbf{case-insensitive} no que toca
a identificadores e palavras reservadas, respeitando a especificação do
Pascal Standard (Ver: \textbf{src/pascal\_analex.py}).

\hypertarget{gestuxe3o-de-palavras-reservadas}{%
\subsection{2.2. Gestão de Palavras
Reservadas}\label{gestuxe3o-de-palavras-reservadas}}

Para evitar que o compilador confunda identificadores (nomes de
variáveis) com comandos da linguagem, utilizámos um dicionário de
mapeamento chamado reserved. - \textbf{Exemplos de Palavras-Chave:}
PROGRAM, BEGIN, END, VAR, IF, THEN, ELSE, WHILE, FOR, FUNCTION,
PROCEDURE, ARRAY, OF, entre outras. - \textbf{Lógica de Identificação:}
Quando o lexer encontra uma sequência de letras, verifica primeiro se
ela existe no dicionário \textbf{reserved}. Se existir, atribui o token
da palavra-chave; caso contrário, classifica-o como um identificador
(ID).

\hypertarget{expressuxf5es-regulares-e-tokens}{%
\subsection{2.3. Expressões Regulares e
Tokens}\label{expressuxf5es-regulares-e-tokens}}

Os tokens foram divididos em duas categorias de definição: 1)
\textbf{Tokens Simples:} Definidos como strings diretas para operadores
e delimitadores. Exemplos: \textbf{t\_PLUS = r'+`\textbf{, }t\_ASSIGN =
r':=`\textbf{, }t\_SEMICOLON = r';'}. 2) \textbf{Tokens Complexos:}
Definidos através de funções para permitir lógica adicional ou capturar
valores. - \textbf{Números:} Distinguimos entre \textbf{NUMBER\_INT} e
\textbf{NUMBER\_REAL} através de padrões que procuram a presença do
ponto decimal. - \textbf{Strings:} O token \textbf{STRING\_LITERAL}
captura texto entre plicas (\textbf{`\ldots{}'}), tratando corretamente
sequências de caracteres.

\hypertarget{tratamento-de-ruuxeddo-e-erros}{%
\subsection{2.4. Tratamento de ``Ruído'' e
Erros}\label{tratamento-de-ruuxeddo-e-erros}}

\begin{itemize}
\tightlist
\item
  \textbf{Espaços e Comentários:} O compilador ignora espaços em branco
  e tabulações através da variável \textbf{t\_ignore}. Além disso,
  implementámos o tratamento de comentários delimitados por chavetas
  \textbf{\{ \ldots{} \}}, garantindo que estas informações não cheguem
  ao analisador sintático.
\item
  \textbf{Rastreamento de Linhas:} A função \textbf{t\_newline}
  incrementa o contador de linhas sempre que encontra um caractere de
  nova linha (**\n**), o que permite emitir mensagens de erro precisas
  com a localização exata no código.
\item
  \textbf{Erros Léxicos:} A função \textbf{t\_error} captura qualquer
  caractere que não corresponda a nenhuma regra definida, interrompendo
  a compilação e informando o utilizador sobre o caractere ilegal
  detetado.
\end{itemize}

    \hypertarget{anuxe1lise-sintuxe1tica-srcparser.py}{%
\section{3. Análise Sintática
(src/parser.py)}\label{anuxe1lise-sintuxe1tica-srcparser.py}}

A análise sintática é a fase onde o compilador verifica se a sequência
de tokens fornecida pelo analisador léxico respeita as regras
gramaticais da linguagem Pascal Standard.

\hypertarget{metodologia-e-ferramentas}{%
\subsection{3.1. Metodologia e
Ferramentas}\label{metodologia-e-ferramentas}}

Utilizámos o \textbf{ply.yacc}, que implementa um algoritmo de análise
\textbf{LALR(1)}. A gramática foi definida através de uma série de
funções Python cujas docstrings contêm as regras de produção na forma de
\textbf{BNF}.

\hypertarget{estrutura-da-gramuxe1tica}{%
\subsection{3.2. Estrutura da
Gramática}\label{estrutura-da-gramuxe1tica}}

A gramática foi desenhada para cobrir a estrutura hierárquica do Pascal:
- \textbf{Raiz do Programa:} A regra p\_program define a estrutura
global: o cabeçalho (program ID;), as declarações e o bloco principal de
comandos BEGIN \ldots{} END.. - \textbf{Declarações:} Implementámos
regras para processar a secção VAR, permitindo a declaração de tipos
simples (integer, real, boolean) e tipos estruturados (array). -
\textbf{Subprogramas:} O parser suporta a definição de procedure e
function, gerindo corretamente a assinatura (parâmetros) e o bloco local
de cada subprograma.

\hypertarget{tratamento-de-expressuxf5es-e-preceduxeancia}{%
\subsection{3.3. Tratamento de Expressões e
Precedência}\label{tratamento-de-expressuxf5es-e-preceduxeancia}}

Para garantir que operações como \textbf{a + b * c} são calculadas
corretamente (a multiplicação antes da soma), definimos níveis de
precedência explicitamente no objeto \textbf{precedence}. Isto resolve
ambiguidades gramaticais sem a necessidade de criar múltiplas sub-regras
complexas para cada operador.

\hypertarget{atributos-sintuxe1ticos-e-auxe7uxf5es}{%
\subsection{3.4. Atributos Sintáticos e
Ações}\label{atributos-sintuxe1ticos-e-auxe7uxf5es}}

Em cada regra gramatical, o analisador executa uma ``ação''. No nosso
projeto: - \textbf{Construção de Expressões:} Cada expressão (expr)
retorna um dicionário contendo o seu \textbf{tipo}, se é uma
\textbf{constante} e o \textbf{código gerado} até ao momento. -
\textbf{Comandos de Controlo:} Regras como \textbf{p\_statement\_if} ou
\textbf{p\_statement\_while} não apenas validam a sintaxe, mas coordenam
a criação de etiquetas de salto (labels) necessárias para o fluxo de
execução na Máquina Virtual.

\hypertarget{recuperauxe7uxe3o-de-erros-sintuxe1ticos}{%
\subsection{3.5. Recuperação de Erros
Sintáticos}\label{recuperauxe7uxe3o-de-erros-sintuxe1ticos}}

Implementámos a função \textbf{p\_error}, que é disparada quando o
parser encontra um token inesperado. Em vez de simplesmente terminar, o
compilador levanta uma exceção \textbf{SyntaxParseError}, indicando ao
utilizador a linha e o token onde a gramática foi violada, facilitando a
correção do código-fonte.

\hypertarget{gramuxe1tica}{%
\subsection{3.6 Gramática}\label{gramuxe1tica}}

\begin{verbatim}
Rule 0     S' -> programa
Rule 1     programa -> PROGRAM ID SEMICOLON bloco DOT
Rule 2     bloco -> decls compound_stmt
Rule 3     decls -> decl decls
Rule 4     decls -> <empty>
Rule 5     decl -> var_section
Rule 6     decl -> subprog_decl
Rule 7     var_section -> VAR var_decl_list
Rule 8     var_decl_list -> var_decl var_decl_list_tail
Rule 9     var_decl_list_tail -> var_decl var_decl_list_tail
Rule 10    var_decl_list_tail -> <empty>
Rule 11    var_decl -> id_list COLON tipo SEMICOLON
Rule 12    id_list -> ID id_list_tail
Rule 13    id_list_tail -> COMMA ID id_list_tail
Rule 14    id_list_tail -> <empty>
Rule 15    tipo -> INTEGER
Rule 16    tipo -> REAL
Rule 17    tipo -> BOOLEAN
Rule 18    tipo -> CHAR
Rule 19    tipo -> STRING
Rule 20    tipo -> array_type
Rule 21    array_type -> ARRAY LBRACKET range RBRACKET OF tipo
Rule 22    range -> NUMBER_INT RANGE NUMBER_INT
Rule 23    subprog_decl -> function_decl
Rule 24    subprog_decl -> procedure_decl
Rule 25    function_header -> FUNCTION ID LPAREN param_list_opt RPAREN COLON tipo SEMICOLON
Rule 26    func_enter -> <empty>
Rule 27    function_decl -> function_header func_enter bloco SEMICOLON
Rule 28    procedure_header -> PROCEDURE ID LPAREN param_list_opt RPAREN SEMICOLON
Rule 29    proc_enter -> <empty>
Rule 30    procedure_decl -> procedure_header proc_enter bloco SEMICOLON
Rule 31    param_list_opt -> param_list
Rule 32    param_list_opt -> <empty>
Rule 33    param_list -> param param_list_tail
Rule 34    param_list_tail -> SEMICOLON param param_list_tail
Rule 35    param_list_tail -> <empty>
Rule 36    param -> id_list COLON tipo
Rule 37    compound_stmt -> BEGIN stmt_list_opt END
Rule 38    stmt_list_opt -> stmt_list
Rule 39    stmt_list_opt -> <empty>
Rule 40    stmt_list -> stmt stmt_list_tail
Rule 41    stmt_list_tail -> SEMICOLON stmt stmt_list_tail
Rule 42    stmt_list_tail -> SEMICOLON
Rule 43    stmt_list_tail -> <empty>
Rule 44    stmt -> assign_stmt
Rule 45    stmt -> if_stmt
Rule 46    stmt -> while_stmt
Rule 47    stmt -> for_stmt
Rule 48    stmt -> repeat_stmt
Rule 49    stmt -> compound_stmt
Rule 50    stmt -> proc_call
Rule 51    assign_stmt -> lvalue ASSIGN expr
Rule 52    var_ref -> ID
Rule 53    var_ref -> ID LBRACKET expr RBRACKET
Rule 54    lvalue -> ID
Rule 55    lvalue -> ID LBRACKET expr RBRACKET
Rule 56    if_stmt -> IF expr THEN stmt
Rule 57    if_stmt -> IF expr THEN stmt ELSE stmt
Rule 58    while_stmt -> WHILE expr DO stmt
Rule 59    for_dir -> TO
Rule 60    for_dir -> DOWNTO
Rule 61    for_stmt -> FOR ID ASSIGN expr for_dir expr DO for_enter stmt for_exit
Rule 62    for_enter -> <empty>
Rule 63    for_exit -> <empty>
Rule 64    repeat_stmt -> REPEAT stmt_list_opt UNTIL expr
Rule 65    proc_call -> ID
Rule 66    proc_call -> ID LPAREN arg_list_opt RPAREN
Rule 67    proc_call -> WRITELN args_opt
Rule 68    proc_call -> READLN read_args_opt
Rule 69    read_args_opt -> LPAREN read_var_list RPAREN
Rule 70    read_args_opt -> <empty>
Rule 71    read_var_list -> lvalue
Rule 72    read_var_list -> lvalue COMMA read_var_list
Rule 73    args_opt -> LPAREN arg_list_opt RPAREN
Rule 74    args_opt -> <empty>
Rule 75    arg_list_opt -> arg_list
Rule 76    arg_list_opt -> <empty>
Rule 77    arg_list -> expr arg_list_tail
Rule 78    arg_list_tail -> COMMA expr arg_list_tail
Rule 79    arg_list_tail -> <empty>
Rule 80    expr -> or_expr
Rule 81    or_expr -> and_expr
Rule 82    or_expr -> or_expr OR and_expr
Rule 83    and_expr -> rel_expr
Rule 84    and_expr -> and_expr AND rel_expr
Rule 85    rel_expr -> add_expr rel_opt
Rule 86    rel_opt -> relop add_expr
Rule 87    rel_opt -> <empty>
Rule 88    relop -> EQUAL
Rule 89    relop -> NOTEQUAL
Rule 90    relop -> LESS
Rule 91    relop -> LESSEQUAL
Rule 92    relop -> GREATER
Rule 93    relop -> GREATEREQUAL
Rule 94    add_expr -> mul_expr
Rule 95    add_expr -> add_expr PLUS mul_expr
Rule 96    add_expr -> add_expr MINUS mul_expr
Rule 97    mul_expr -> unary_expr
Rule 98    mul_expr -> mul_expr TIMES unary_expr
Rule 99    mul_expr -> mul_expr DIVIDE unary_expr
Rule 100   mul_expr -> mul_expr DIV unary_expr
Rule 101   mul_expr -> mul_expr MOD unary_expr
Rule 102   unary_expr -> MINUS unary_expr
Rule 103   unary_expr -> NOT unary_expr
Rule 104   unary_expr -> primary
Rule 105   primary -> NUMBER_REAL
Rule 106   primary -> NUMBER_INT
Rule 107   primary -> STRING_LITERAL
Rule 108   primary -> TRUE
Rule 109   primary -> FALSE
Rule 110   primary -> var_ref
Rule 111   primary -> ID LPAREN arg_list_opt RPAREN
Rule 112   primary -> LPAREN expr RPAREN
\end{verbatim}

    \hypertarget{anuxe1lise-semuxe2ntica-e-gestuxe3o-de-contexto-srcsem.py-e-srccontext.py}{%
\section{4. Análise Semântica e Gestão de Contexto (src/sem.py e
src/context.py)}\label{anuxe1lise-semuxe2ntica-e-gestuxe3o-de-contexto-srcsem.py-e-srccontext.py}}

A análise semântica é responsável por verificar se o programa, embora
sintaticamente correto, faz sentido do ponto de vista da lógica da
linguagem Pascal.

\hypertarget{tabela-de-suxedmbolos-e-gestuxe3o-de-escopos}{%
\subsection{4.1. Tabela de Símbolos e Gestão de
Escopos}\label{tabela-de-suxedmbolos-e-gestuxe3o-de-escopos}}

Utilizámos uma estrutura de Tabela de Símbolos implementada em sem.py,
que funciona através de uma pilha de dicionários (scopes). -
\textbf{Escopo Global vs.~Local:} Ao entrar numa função ou procedimento,
o compilador faz um \textbf{push} de um novo dicionário para a pilha,
criando um novo escopo. Ao sair, faz um \textbf{pop}. Isto permite que
variáveis locais existam apenas durante a execução do subprograma,
evitando conflitos com variáveis globais. - \textbf{Deteção de
Identificadores:} O método \textbf{lookup} percorre a pilha de escopos
do topo (mais local) para a base (global), garantindo que o compilador
encontra sempre a instância mais próxima de uma variável.

\hypertarget{verificauxe7uxe3o-de-tipos-type-checking}{%
\subsection{4.2. Verificação de Tipos (Type
Checking)}\label{verificauxe7uxe3o-de-tipos-type-checking}}

O compilador realiza verificações estritas para garantir a integridade
dos dados: - \textbf{Compatibilidade Aritmética:} Implementámos a lógica
que permite operações entre \textbf{integer} e \textbf{real} (resultando
em \textbf{real}), mas impede operações inválidas entre tipos
incompatíveis. - \textbf{Arrays:} Validamos se o índice utilizado para
aceder a um array é do tipo \textbf{integer}. Além disso, o sistema
verifica a consistência entre o tipo base do array e o valor que está a
ser atribuído. - \textbf{Chamadas de Funções:} O módulo \textbf{sem.py}
verifica se o número e os tipos de argumentos passados numa chamada
correspondem exatamente à assinatura definida na declaração da função.

\hypertarget{gestuxe3o-de-memuxf3ria-e-contexto-context.py}{%
\subsection{4.3. Gestão de Memória e Contexto
(context.py)}\label{gestuxe3o-de-memuxf3ria-e-contexto-context.py}}

O ficheiro \textbf{context.py} atua como um repositório central de
estado durante a compilação: - \textbf{Alocação de Endereços:} Gere
contadores automáticos para endereços globais
(\textbf{next\_global\_addr}) e locais
(\textbf{next\_local\_addr\_stack}), garantindo que cada variável ocupa
um slot único na stack da VM. - \textbf{Tratamento de Built-ins:} O
sistema pré-declara funções essenciais (como \textbf{writeln},
\textbf{readln}, \textbf{length}) na tabela de símbolos global,
impedindo que o utilizador as redeclare acidentalmente.

\hypertarget{tratamento-de-erros-semuxe2nticos}{%
\subsection{4.4. Tratamento de Erros
Semânticos}\label{tratamento-de-erros-semuxe2nticos}}

Sempre que uma regra é violada (ex: usar uma variável x que não foi
declarada em \textbf{VAR}), o compilador levanta uma
\textbf{SemanticError}. Esta exceção interrompe o pipeline e fornece ao
utilizador uma mensagem clara com a descrição do problema e a linha onde
ocorreu.

    \hypertarget{gerauxe7uxe3o-de-cuxf3digo-srccodegen.py}{%
\section{5. Geração de Código
(src/codegen.py)}\label{gerauxe7uxe3o-de-cuxf3digo-srccodegen.py}}

A geração de código é a fase final (back-end) onde as estruturas
validadas são convertidas em instruções de baixo nível para a Máquina
Virtual (VM) de stack.

\hypertarget{modelo-de-execuuxe7uxe3o-baseada-em-stack}{%
\subsection{5.1. Modelo de Execução baseada em
Stack}\label{modelo-de-execuuxe7uxe3o-baseada-em-stack}}

O código gerado assume uma arquitetura de stack, onde as operações
retiram os operandos do topo da pilha e colocam o resultado no mesmo
local. - \textbf{Instruções Aritméticas:} Operações como \textbf{a + b}
são traduzidas para uma sequência de carregamento (\textbf{PUSH})
seguida da instrução de operação (\textbf{ADD}, \textbf{SUB},
\textbf{MUL}, \textbf{DIV}). - \textbf{Conversão de Tipos:} Quando o
analisador semântico deteta uma operação entre um inteiro e um real, o
gerador emite a instrução \textbf{ITOF} para garantir a coerência dos
dados na stack.

\hypertarget{mapeamento-de-memuxf3ria-e-endereuxe7amento}{%
\subsection{5.2. Mapeamento de Memória e
Endereçamento}\label{mapeamento-de-memuxf3ria-e-endereuxe7amento}}

A gestão de variáveis é feita distinguindo o seu escopo: -
\textbf{Variáveis Globais:} Utilizam as instruções \textbf{PUSHG} e
\textbf{STOREG}, apontando para endereços absolutos definidos no início
da compilação através do \textbf{next\_global\_addr}. -
\textbf{Variáveis Locais:} Utilizam \textbf{PUSHL} e \textbf{STOREL},
com endereços relativos à base da frame atual da stack (FP), geridos
pelo \textbf{next\_local\_addr\_stack} no \textbf{context.py}.

\hypertarget{implementauxe7uxe3o-de-estruturas-de-controlo}{%
\subsection{5.3. Implementação de Estruturas de
Controlo}\label{implementauxe7uxe3o-de-estruturas-de-controlo}}

Para implementar saltos e ciclos, o módulo codegen.py fornece um gerador
de etiquetas únicas (new\_label). - \textbf{Condicionais (IF):} O
gerador emite um \textbf{JZ} (jump if zero) para saltar o bloco
\textbf{THEN} caso a condição seja falsa. Se existir um \textbf{ELSE}, é
gerado um \textbf{JUMP} no final do bloco \textbf{THEN} para evitar a
execução do código alternativo. - \textbf{Ciclos (WHILE/FOR):} São
criadas etiquetas no início (para repetição) e no fim (para saída). No
caso do ciclo \textbf{FOR}, o compilador gera automaticamente o código
de incremento/decremento da variável de controlo e a verificação do
limite.

\hypertarget{chamadas-de-subprogramas-funuxe7uxf5es-e-procedimentos}{%
\subsection{5.4. Chamadas de Subprogramas (Funções e
Procedimentos)}\label{chamadas-de-subprogramas-funuxe7uxf5es-e-procedimentos}}

A geração de código para subprogramas segue um protocolo rigoroso: 1)
\textbf{Ativação:} É gerada a etiqueta com o nome da função (ou
\textbf{label} único). 2) \textbf{Passagem de Parâmetros:} O código
coloca os argumentos na stack antes da instrução \textbf{CALL}. 3)
\textbf{Retorno de Valores:} Para funções, o compilador reserva um slot
na stack (\textbf{push\_default\_for\_type}) para o valor de retorno
antes de empilhar os argumentos, garantindo que o resultado fica
disponível para o chamador após o \textbf{POP} dos argumentos.

\hypertarget{strings-e-io}{%
\subsection{5.5. Strings e I/O}\label{strings-e-io}}

\begin{itemize}
\tightlist
\item
  \textbf{Strings:} O compilador gera instruções PUSHS para literais de
  texto.
\item
  \textbf{Input/Output:} As chamadas a \textbf{readln} e
  \textbf{writeln} são traduzidas para instruções nativas da VM
  (\textbf{READ}, \textbf{WRITE}, \textbf{WRITELN}), permitindo a
  interação com o utilizador.
\end{itemize}

    \hypertarget{testes-e-resultados}{%
\section{6. Testes e Resultados}\label{testes-e-resultados}}

A validação do compilador foi realizada através de uma infraestrutura de
testes automatizada, desenhada para garantir a fiabilidade da geração de
código e a precisão dos diagnósticos de erro.

\hypertarget{infraestrutura-de-testes-automuxe1tica}{%
\subsection{6.1. Infraestrutura de Testes
Automática}\label{infraestrutura-de-testes-automuxe1tica}}

Afastando-nos de uma abordagem de testes manuais, implementámos o script
\textbf{run\_tests.py}. Este motor de testes percorre sistematicamente a
diretoria \textbf{tests/cases/}, executando o pipeline para cada
ficheiro \textbf{.pas} e comparando o resultado com as expetativas
definidas no sistema.

\hypertarget{testes-de-regressuxe3o-casos-de-sucesso}{%
\subsection{6.2. Testes de Regressão (Casos de
Sucesso)}\label{testes-de-regressuxe3o-casos-de-sucesso}}

Na pasta \textbf{tests/cases/ok/}, mantemos uma suite de testes que
cobre todas as funcionalidades exigidas no enunciado: -
\textbf{Expressões e Atribuições:} Testes de precedência aritmética e
lógica. - \textbf{Estruturas de Controlo:} Validação de ciclos
\textbf{FOR} (tanto \textbf{to} como \textbf{downto}), \textbf{WHILE} e
condicionais \textbf{IF-THEN-ELSE}. - \textbf{Subprogramas:} Chamadas de
funções e procedimentos com passagem de parâmetros e gestão de valor de
retorno. - \textbf{Estruturas de Dados:} Manipulação de arrays
(incluindo acesso indexado) e strings. - \textbf{Exemplos do Enunciado:}
Os cinco exemplos propostos (Olá Mundo, Fatorial, Números Primos, Soma
de Array e Binário para Inteiro) são compilados e os resultados em
assembly são validados na VM.

\hypertarget{testes-de-stress-e-diagnuxf3stico-casos-de-erro}{%
\subsection{6.3. Testes de Stress e Diagnóstico (Casos de
Erro)}\label{testes-de-stress-e-diagnuxf3stico-casos-de-erro}}

Um dos pontos fortes deste compilador é a sua capacidade de lidar com
código inválido. Através do ficheiro
\textbf{tests/manifests/error\_cases.json}, definimos um manifesto que
mapeia ficheiros de erro (na pasta \textbf{cases/error/}) às respetivas
mensagens que o compilador deve emitir.

O sistema valida se o compilador deteta erros como: - \textbf{Violações
de Escopo:} Uso de variáveis não declaradas (ex: \textbf{T1.pas}). -
\textbf{Incompatibilidade de Tipos:} Operações inválidas, como o uso do
operador \textbf{MOD} com tipos reais ou condições de \textbf{IF} que
não resultam em boolean (ex: \textbf{T11.pas}, \textbf{T19.pas}). -
\textbf{Erros de Assinatura:} Chamadas de funções com número ou tipos de
argumentos incorretos (ex: \textbf{T28.pas}, \textbf{T29.pas}).

\hypertarget{anuxe1lise-de-resultados}{%
\subsection{6.4. Análise de Resultados}\label{anuxe1lise-de-resultados}}

O sucesso na execução da suite completa de testes demonstra que o
compilador possui: 1) \textbf{Estabilidade:} Alterações em módulos (como
o \textbf{codegen.py}) não quebram funcionalidades anteriores. 2)
\textbf{Precisão:} As mensagens de erro são específicas, indicando
exatamente a falha lógica (ex: ``Comparação inválida'' ou ``Range
inválido''). 3) \textbf{Conformidade:} O código da VM gerado respeita a
stack discipline e os limites de memória definidos pela arquitetura de
destino.

    \hypertarget{conclusuxe3o-e-dificuldades-superadas}{%
\section{7. Conclusão e Dificuldades
Superadas}\label{conclusuxe3o-e-dificuldades-superadas}}

O desenvolvimento deste compilador para Pascal Standard representou um
desafio técnico significativo, exigindo a integração harmoniosa de
conceitos teóricos de linguagens formais com soluções práticas de
engenharia de software.

\hypertarget{suxedntese-do-trabalho-realizado}{%
\subsection{7.1. Síntese do Trabalho
Realizado}\label{suxedntese-do-trabalho-realizado}}

Conseguimos implementar um sistema funcional que cobre desde a análise
de texto bruto até à geração de código executável. A modularização em
Python permitiu que cada fase (Léxica, Sintática, Semântica e Geração)
fosse testada de forma independente, resultando num produto final
robusto que cumpre todos os requisitos do enunciado, incluindo o suporte
a subprogramas e tipos estruturados como \textbf{array} (Ver:
\textbf{src/compiler.py}).

\hypertarget{dificuldades-e-soluuxe7uxf5es-tuxe9cnicas}{%
\subsection{7.2. Dificuldades e Soluções
Técnicas}\label{dificuldades-e-soluuxe7uxf5es-tuxe9cnicas}}

Durante o desenvolvimento, enfrentámos desafios complexos que exigiram
soluções criativas: - \textbf{Gestão da Stack Discipline:} Um dos
maiores desafios foi garantir que a stack da VM permanecesse equilibrada
após chamadas de funções. A solução passou por reservar antecipadamente
um slot para o valor de retorno (\textbf{push\_default\_for\_type}) e
limpar os argumentos (\textbf{POP k}) logo após o retorno do subprograma
(Ver: \textbf{src/codegen.py}). - \textbf{Compatibilidade de Tipos
(Coerção):} A mistura de \textbf{integer} e \textbf{real} em expressões
aritméticas exigiu a inserção automática da instrução \textbf{ITOF}. A
implementação desta lógica no Middle-end permitiu que o Back-end gerasse
código correto sem intervenção do utilizador (Ver:
\textbf{src/parser.py}). - \textbf{Gestão de Identificadores
Reservados:} Para impedir que utilizadores redeclarassem funções
internas (como \textbf{writeln} ou \textbf{length}), implementámos um
mecanismo de proteção na Tabela de Símbolos que bloqueia qualquer
tentativa de shadowing de nomes built-in (Ver: \textbf{src/sem.py}).

\hypertarget{melhorias}{%
\subsection{7.3. Melhorias}\label{melhorias}}

Embora o compilador esteja totalmente funcional, a arquitetura modular
adotada permite futuras expansões, tais como: - \textbf{Otimização de
Código:} Implementação de Constant Folding (cálculo de expressões
constantes em tempo de compilação). - \textbf{Novos Tipos:} Expansão
para suportar \textbf{records} ou \textbf{pointers}. - \textbf{Geração
de Código Nativo:} Adaptar o Back-end para gerar instruções x86 ou ARM.

    \begin{tcolorbox}[breakable, size=fbox, boxrule=1pt, pad at break*=1mm,colback=cellbackground, colframe=cellborder]
\prompt{In}{incolor}{ }{\boxspacing}
\begin{Verbatim}[commandchars=\\\{\}]

\end{Verbatim}
\end{tcolorbox}


    % Add a bibliography block to the postdoc
    
    
    
\end{document}
